% !TEX program = xelatex
\documentclass[aspectratio=169,11pt]{beamer}

% ── 中文支持 ──
\usepackage{fontspec}
\usepackage[UTF8,noindent]{ctex}

% ── 主题 ──
\usetheme{Madrid}
\usecolortheme{whale}
\setbeamertemplate{navigation symbols}{}
\setbeamertemplate{footline}[frame number]

% ── 包 ──
\usepackage{graphicx}
\usepackage{array}
\usepackage{booktabs}
\usepackage{tikz}
\usepackage{listings}
\usepackage{xcolor}
\usepackage{fontawesome5}

\usetikzlibrary{arrows.meta,positioning,shapes.geometric,fit,calc}

% ── 颜色 ──
\definecolor{accent}{HTML}{2E86C1}
\definecolor{good}{HTML}{27AE60}
\definecolor{bad}{HTML}{E74C3C}
\definecolor{codebg}{HTML}{F5F5F5}
\definecolor{codeframe}{HTML}{DDDDDD}

\setbeamercolor{block title}{bg=accent,fg=white}
\setbeamercolor{block body}{bg=accent!8}

% ── 代码样式 ──
\lstset{
  basicstyle=\ttfamily\scriptsize,
  backgroundcolor=\color{codebg},
  frame=single,
  rulecolor=\color{codeframe},
  breaklines=true,
  columns=fullflexible,
  language=Python,
  keywordstyle=\color{accent}\bfseries,
  commentstyle=\color{gray},
  stringstyle=\color{good},
  showstringspaces=false,
}

% ── 元信息 ──
\title[Arithmetic Helper]{Arithmetic Practice Assistant}
\subtitle{面向儿童的智能算术练习桌面应用}
\author{Development Team}
\date{\today}

\begin{document}

% ============================
% 封面
% ============================
\begin{frame}
  \titlepage
\end{frame}

% ============================
% 目录
% ============================
\begin{frame}{Outline}
  \tableofcontents
\end{frame}

% ============================================================
\section{Sales Pitch}
% ============================================================

% ------- 产品定位 -------
\begin{frame}{产品定位}
  \begin{columns}[T]
    \column{0.55\textwidth}
    \begin{block}{我们是什么?}
      一款面向 \textbf{K-12 儿童} 的桌面算术练习工具:
      \begin{itemize}
        \item 可配置的四则运算 \& 混合运算
        \item 支持括号、难度梯度
        \item \textbf{手写识别}输入
        \item 即时反馈 + 成绩统计
      \end{itemize}
    \end{block}

    \column{0.4\textwidth}
    \centering
    \begin{tikzpicture}[scale=0.8]
      \node[draw,rounded corners=8pt,fill=accent!12,
            minimum width=3.6cm,minimum height=2.8cm,align=center]
        {\Large\faCalculator\\[6pt]\small 简洁 · 高效 · 有趣};
    \end{tikzpicture}
  \end{columns}
\end{frame}

% ------- 核心卖点一览 -------
\begin{frame}{核心卖点}
  \begin{columns}[T]
    \column{0.32\textwidth}
    \centering
    {\Large\faLaptopCode}\\[6pt]
    \textbf{简洁界面}\\[4pt]
    \small 儿童友好大字体($\ge$14pt)\\
    Material Design 风格\\
    中英文一键切换

    \column{0.32\textwidth}
    \centering
    {\Large\faPenFancy}\\[6pt]
    \textbf{流畅手写识别}\\[4pt]
    \small 画布手写 → 即时识别\\
    多后端可选\\
    本地/云端灵活切换

    \column{0.32\textwidth}
    \centering
    {\Large\faChartBar}\\[6pt]
    \textbf{丰富历史记录}\\[4pt]
    \small 按姓名检索\\
    正确率颜色标识\\
    逐题详情回顾
  \end{columns}
\end{frame}

% ------- 成本对比 -------
\begin{frame}{方案对比:为什么不用 AI 出题?}
  \centering
  \renewcommand{\arraystretch}{1.4}
  \begin{tabular}{l >{\centering\arraybackslash}p{5cm} >{\centering\arraybackslash}p{5cm}}
    \toprule
    & \textbf{本项目(代码生成)} & \textbf{纯 AI 生成} \\
    \midrule
    \textbf{正确性}
      & \textcolor{good}{\faCheckCircle\; 100\% 确定性正确}
      & \textcolor{bad}{\faTimesCircle\; 可能出错} \\
    \textbf{成本}
      & \textcolor{good}{\faCheckCircle\; 零边际成本}
      & \textcolor{bad}{\faTimesCircle\; 按 token 计费} \\
    \textbf{延迟}
      & \textcolor{good}{\faCheckCircle\; 毫秒级本地生成}
      & \textcolor{bad}{\faTimesCircle\; 网络延迟} \\
    \textbf{离线可用}
      & \textcolor{good}{\faCheckCircle\; 完全离线}
      & \textcolor{bad}{\faTimesCircle\; 依赖网络} \\
    \bottomrule
  \end{tabular}

  \vspace{12pt}
  \begin{block}{结论}
    对于\textbf{结构化、规则明确}的算术题目,代码生成在正确性和成本上\textbf{完胜} AI。
  \end{block}
\end{frame}

% ------- 手写识别 -------
\begin{frame}{手写识别:多后端策略}
  \begin{columns}[T]
    \column{0.5\textwidth}
    \begin{block}{本地后端(零成本)}
      \begin{itemize}
        \item \texttt{sklearn-svm} — 内置 SVM 分类器
        \item \texttt{tesseract} — Tesseract OCR
        \item \texttt{paddle-ocr} — PaddleOCR
      \end{itemize}
    \end{block}

    \column{0.5\textwidth}
    \begin{block}{云端后端(高精度)}
      \begin{itemize}
        \item Google Cloud Vision
        \item 百度 OCR
        \item 腾讯云 OCR
      \end{itemize}
    \end{block}
  \end{columns}

  \vspace{10pt}
  \centering
  \tikz{
    \node[draw=accent,rounded corners,fill=accent!8,
          inner sep=8pt,text width=12cm,align=center]
      {用户可在设置页面自由选择后端 — 离线环境用本地,追求精度用云端};
  }
\end{frame}

% ------- 历史记录功能 -------
\begin{frame}{历史记录功能}
  \begin{columns}[T]
    \column{0.55\textwidth}
    \begin{itemize}
      \item 每次练习自动保存到 CSV
      \item 按学生姓名筛选
      \item 正确率颜色编码:
        \begin{itemize}
          \item \textcolor{good}{$\ge 80\%$:绿色}
          \item \textcolor{orange}{$\ge 60\%$:黄色}
          \item \textcolor{bad}{$< 60\%$:红色}
        \end{itemize}
      \item 点击查看逐题详情
      \item 便于家长/教师追踪学习进度
    \end{itemize}

    \column{0.4\textwidth}
    \centering
    \begin{tikzpicture}[scale=0.75,every node/.style={font=\small}]
      \node[draw,rounded corners,fill=good!15,minimum width=3cm,minimum height=0.7cm]
        (r1) at (0,2) {Session A — 95\%};
      \node[draw,rounded corners,fill=orange!15,minimum width=3cm,minimum height=0.7cm]
        (r2) at (0,1) {Session B — 72\%};
      \node[draw,rounded corners,fill=bad!15,minimum width=3cm,minimum height=0.7cm]
        (r3) at (0,0) {Session C — 48\%};
      \node[above=4pt of r1,font=\bfseries] {历史列表};
    \end{tikzpicture}
  \end{columns}
\end{frame}

% ============================================================
\section{开发与代码}
% ============================================================

% ------- 项目架构总览 -------
\begin{frame}{面向对象分层架构}
  \centering
  \begin{tikzpicture}[
    layer/.style={draw,rounded corners=6pt,minimum width=10cm,
                  minimum height=1cm,align=center,font=\small\bfseries},
    arr/.style={-{Stealth[length=5pt]},thick,gray},
    node distance=0.5cm
  ]
    \node[layer,fill=accent!60,text=white]   (ui)   {UI 层 \normalfont — pages / widgets / main\_window};
    \node[layer,fill=accent!40,text=white,below=of ui]  (ctrl) {Controller 层 \normalfont — practice\_controller};
    \node[layer,fill=accent!25,below=of ctrl] (svc)  {Service 层 \normalfont — problem\_generator / session\_service / recognizers};
    \node[layer,fill=accent!12,below=of svc]  (repo) {Repository 层 \normalfont — history\_repository (CSV)};
    \node[layer,fill=gray!12,below=of repo]   (dom)  {Domain 层 \normalfont — models (dataclasses)};

    \draw[arr] (ui)   -- (ctrl);
    \draw[arr] (ctrl) -- (svc);
    \draw[arr] (svc)  -- (repo);
    \draw[arr] (repo) -- (dom);
    \draw[arr] (svc)  -- (dom);
  \end{tikzpicture}
\end{frame}

% ------- 目录结构 -------
\begin{frame}[fragile]{项目目录结构}
\begin{lstlisting}[language={},basicstyle=\ttfamily\scriptsize]
arithmetic_helper/
  main.py                        # 入口 + .env 加载
  app/
    domain/models.py             # PracticeConfig, PracticeQuestion,
                                 # AnswerRecord, SessionResult
    services/
      recognizer_backend.py      # RecognizerBackend (ABC)
      handwriting_recognizer.py  # sklearn SVM
      google_vision_recognizer.py
      baidu_ocr_recognizer.py
      tencent_ocr_recognizer.py
      tesseract_recognizer.py
      paddle_ocr_recognizer.py
      problem_generator.py       # 题目生成引擎
      session_service.py         # 会话生命周期管理
    repositories/
      history_repository.py      # CSV 持久化
    controllers/
      practice_controller.py     # 信号驱动的 UI 编排
    ui/
      main_window.py             # 主窗口 + 依赖注入
      pages/  widgets/           # 页面 & 组件
    i18n/localizer.py            # 国际化
\end{lstlisting}
\end{frame}

% ------- 设计模式 -------
\begin{frame}{关键设计模式}
  \begin{columns}[T]
    \column{0.48\textwidth}
    \begin{block}{\faProjectDiagram\; 策略模式(Strategy)}
      \texttt{RecognizerBackend} 抽象基类\\
      6 种具体实现可即插即用\\
      运行时按用户选择切换
    \end{block}

    \vspace{6pt}
    \begin{block}{\faIndustry\; 工厂模式(Factory)}
      \texttt{MainWindow.\_build\_recognizer()}\\
      根据 key 字符串实例化后端\\
      内置缓存避免重复创建
    \end{block}

    \column{0.48\textwidth}
    \begin{block}{\faSyringe\; 依赖注入(DI)}
      \texttt{MainWindow} 统一创建并注入\\
      Service → Controller → Page\\
      方便单元测试 mock
    \end{block}

    \vspace{6pt}
    \begin{block}{\faBroadcastTower\; 信号/槽通信}
      PyQt 信号驱动 UI 更新\\
      Controller 发射信号\\
      Page 订阅响应
    \end{block}
  \end{columns}
\end{frame}

% ------- 识别抽象类 -------
\begin{frame}[fragile]{识别后端:抽象与扩展}
\begin{lstlisting}[language=Python]
class RecognizerBackend(ABC):
    @abstractmethod
    def recognize(self, image: QImage) -> int | None:
        """识别手写内容,返回整数或 None"""

    @property
    @abstractmethod
    def name(self) -> str:
        """后端显示名称"""

    @property
    @abstractmethod
    def available(self) -> bool:
        """依赖/密钥是否就绪"""

    @staticmethod
    def _qimage_to_png_bytes(image: QImage) -> bytes | None:
        """共享工具方法:QImage -> PNG bytes"""
\end{lstlisting}

  \vspace{4pt}
  \centering
  \tikz{\node[draw=good,rounded corners,fill=good!8,inner sep=6pt,text width=11cm,align=center]
    {扩展新后端只需继承此类,实现 3 个抽象成员即可接入系统};}
\end{frame}

% ------- API Key 管理 -------
\begin{frame}[fragile]{通用 API-Key 接口}
  \begin{columns}[T]
    \column{0.5\textwidth}
    \begin{block}{机制}
      \begin{enumerate}
        \item \texttt{main.py} 启动时加载 \texttt{.env}
        \item 写入 \texttt{os.environ}
        \item 各 Recognizer 读取所需变量
        \item \texttt{available} 属性检查密钥
      \end{enumerate}
    \end{block}

    \column{0.48\textwidth}
\begin{lstlisting}[language={}]
# .env 示例
GOOGLE_VISION_API_KEY=xxx
BAIDU_API_KEY=xxx
BAIDU_SECRET_KEY=xxx
TENCENT_SECRET_ID=xxx
TENCENT_SECRET_KEY=xxx
\end{lstlisting}
  \end{columns}

  \vspace{8pt}
  \centering
  \tikz{\node[draw=accent,rounded corners,fill=accent!8,
    inner sep=6pt,text width=12cm,align=center]{
    用户只需修改 \texttt{.env} 文件即可切换 OCR 提供商 — \textbf{零代码改动}
  };}
\end{frame}

% ------- 技术栈 -------
\begin{frame}{技术栈}
  \centering
  \renewcommand{\arraystretch}{1.4}
  \begin{tabular}{ll}
    \toprule
    \textbf{层面} & \textbf{技术} \\
    \midrule
    GUI 框架       & PyQt5 + qt-material 主题 \\
    语言           & Python 3.10+ \\
    本地识别       & scikit-learn (SVM)、Tesseract、PaddleOCR \\
    云端识别       & Google Vision、百度 OCR、腾讯云 OCR \\
    数据持久化     & CSV(轻量、可迁移) \\
    国际化         & 自研 Localizer(zh\_CN / en\_US) \\
    样式           & Material Design + 自定义 CSS \\
    \bottomrule
  \end{tabular}
\end{frame}

% ------- 可扩展性 -------
\begin{frame}{可扩展性}
  \begin{columns}[T]
    \column{0.48\textwidth}
    \begin{block}{当前支持}
      \begin{itemize}
        \item 四则运算 + 混合运算
        \item 6 种识别后端
        \item 中文 / 英文界面
        \item CSV 历史存储
      \end{itemize}
    \end{block}

    \column{0.48\textwidth}
    \begin{block}{扩展方向}
      \begin{itemize}
        \item 新题型(分数、方程 \dots)
        \item 新识别后端(继承 ABC)
        \item 新语言(扩展字典)
        \item 新存储(SQLite / 云端)
        \item 自适应难度算法
      \end{itemize}
    \end{block}
  \end{columns}

  \vspace{10pt}
  \centering
  \begin{tikzpicture}
    \node[draw=accent,rounded corners=10pt,fill=accent!8,
          inner sep=10pt,text width=11cm,align=center,font=\large]
      {分层解耦 + 抽象接口 = \textbf{低成本扩展}};
  \end{tikzpicture}
\end{frame}

% ============================================================
\section{总结}
% ============================================================

\begin{frame}{总结}
  \begin{enumerate}
    \setlength{\itemsep}{10pt}
    \item \textbf{产品层面}:简洁 UI + 手写识别 + 丰富历史 → 儿童友好的练习体验
    \item \textbf{成本优势}:代码确定性生成题目,零边际成本,100\% 正确
    \item \textbf{工程质量}:五层解耦架构、策略/工厂/DI 设计模式
    \item \textbf{灵活扩展}:抽象识别接口 + 通用 API-Key 管理 → 即插即用
  \end{enumerate}

  \vspace{16pt}
  \centering
  {\Large\bfseries\textcolor{accent}{Thank You!}}\\[6pt]
  {\small Questions \& Discussion}
\end{frame}

\end{document}
